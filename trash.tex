

% Para cumplir el propósito de la tarea, se brindó un conjunto de datos
% obtenido de Twitter, el cual comprende tres idiomas: árabe, inglés y español.
% La tarea permitía la clasificación para un subconjunto de idiomas; usando sólo texto, imágenes o haciendo una fusión de las dos fuentes de información.
% A través de una extracción del contenido
% semántico de las imágenes de cada usuario podemos entrenar un modelo
% de manera eficiente y eficaz, para cada posible subconjunto de imágenes compartidas por los usuarios sean
% independientes del idioma permite utilizar
% un modelo, para clasificación de género, entrenado con imágenes de usuarios 
% en un idioma diferente al idioma de los usuarios sobre el que se desea aplicar.
% La principal contribución de esta investigación es la confirmación
% de que, para la tarea de identificación de género, las imágenes son independientes del
% idioma de los usuarios que las comparten. 

% - En relación a nuestro problema o enfoque de solución, creo que falta
% comentar la idea o hipótesis que tenemos, y que de alguna manera
% pretendemos probar: algo como que personas con rasgos demográficos
% similares, aunque de distintas lenguas y/o culturas, suelen compartir
% imágenes semejantes, que contienen mas o menos los mismos objetos. Tal
% vez valdría la pena cerrar con un por ejemplo: los hombres, sin importar
% su lengua, suelen compartir más imágenes de X que las mujeres, y por el
% contrario, las mujeres mas imágenes de Y que los hombres.



% usuarios [algunas citas]; cerrar introduciendo el problema que nosotros
% estamos abordando.




% Existen varias conferencias cuya meta 
% es promover la investigación en áreas de recuperación
% de información. Entre ellas están la Text REtrieval Conference (TREC) y la 
% iniciativa CLEF (Conference and Labs of the Evaluation Forum, anteriormente 
% conocida como Cross-Language Evaluation Forum). Dentro de esta última, se 
% encuentra el PAN, una serie de eventos científicos y tareas sobre textos 
% forenses y estilometría de textos digitales.
% %  \cite{pan}. 

% El objetivo de la tarea
% de perfilado de autores del PAN 2018 fue abordar la identificación de género
% desde una perspectiva multimodal donde además de texto se
% incluyeron imágenes \cite{rangel_rosso_montes-y-gomez_potthast_stein}.
% Para cumplir el propósito de la tarea, se brindó un conjunto de datos
% obtenido de Twitter, el cual comprende tres idiomas: árabe, inglés y español.
% La tarea permitía la clasificación para un subconjunto de idiomas; usando sólo texto, imágenes o haciendo una fusión de las dos fuentes de información.



% A partir de los resultados obtenidos usando sólo imágenes 
%  \cite{rangel_rosso_montes-y-gomez_potthast_stein}, se pudo observar que es posible 
% utilizar información de éstas para el perfilado de usuarios.
% La hipótesis de este trabajo es que las imágenes de una colección para un
% idioma  pueden ser útiles para identificar el perfil en otro idioma. 
% Lo anterior es útil cuando no se cuenta con la cantidad suficiente de datos
% para un grupo de usuarios en un idioma, pero sí para otros idiomas. Por ejemplo,
% en la tarea de perfilado de autores del PAN, se cuentan con el doble de usuarios
% en inglés y español en comparación a los del idioma árabe.

% La principal contribución de esta investigación es la confirmación
% de que, para la tarea de identificación de género, las imágenes son independientes del
% idioma de los usuarios que las comparten. 

% A través de una extracción del contenido
% semántico de las imágenes de cada usuario podemos entrenar un modelo
% de manera eficiente y eficaz, para cada posible subconjunto de imágenes compartidas por los usuarios sean
% independientes del idioma permite utilizar
% un modelo, para clasificación de género, entrenado con imágenes de usuarios 
% en un idioma diferente al idioma de los usuarios sobre el que se desea aplicar.




% \begin{itemize}
%     \item Describir la tarea de perfilado de autores.
%     \item Predicción de género, ¿por qué? Marketing, otra motivación
%     sería mejor.
%     \item Uso de texto e imágenes para perfilado. Mencionar que 
%     utilizaré imágenes.
%     \item Hablar sobre los foros de evaluación, PAN 2018. Describir la
%     tarea.
%     \item Hablar de los resultados obtenidos por Takahashi y Ezra. El
%     trabajo del último sirve como base para mi proyecto.
%     \item Describir el problema de la clasificación de género 
%     usando imágenes de usuarios en otro idioma, ¿cómo la atacaré?,
%     ¿el idioma importa?, ¿es un problema el tamaño de los conjuntos de 
%     datos en un idioma?
%     \item Contribuciones.
% \end{itemize}